% !TEX TS-program = pdflatexmk
\documentclass[12pt]{article}

% Layout.
\usepackage[top=.75in, bottom=0.75in, left=.75in, right=.75in, headheight=1in, headsep=6pt]{geometry}

% Fonts.
\usepackage{mathptmx}
\usepackage[scaled=0.86]{helvet}
\renewcommand{\emph}[1]{\textsf{\textbf{#1}}}

% Misc packages.
\usepackage{amsmath,amssymb,latexsym}
\usepackage{graphicx,tikz}
\usepackage{array}
\usepackage{xcolor}
\usepackage{multicol}
\usepackage{tabularx,colortbl}
\usepackage{enumitem}
%to make tikz pics work
\usepackage{tikz,pgfplots}
\usetikzlibrary{arrows}
\newcommand{\midarrow}{\tikz \draw[-triangle 90] (0,0) -- +(.1,0);}

\usepackage[colorlinks=true]{hyperref}

% Paragraph spacing
\parindent 0pt
\parskip 6pt plus 1pt
\def\tableindent{\hskip 0.5 in}
\def\ts{\hskip 1.5 em}

\usepackage{fancyhdr}
\pagestyle{fancy} 
\lhead{\large\sf\textbf{MATH 663 }}
\rhead{\large\sf\textbf{Fall 2023}}
\chead{\large\sf\textbf{HW 7 }}

\newcommand{\localhead}[1]{\par\smallskip\textbf{#1}\nobreak\\}%
\def\heading#1{\localhead{\large\emph{#1}}}
\def\subheading#1{\localhead{\emph{#1}}}

%% Special Math Symbol shortcuts
\newcommand{\bbN}{\mathbb{N}}
\newcommand{\rad}{\text{rad}}
\newcommand{\diam}{\text{diam}}

%\newenvironment{clist}%
%{\bgroup\parskip 0pt\begin{list}{$\bullet$}{\partopsep 4pt\topsep 0pt\itemsep -2pt}}%
%{\end{list}\egroup}%

\usetikzlibrary{calc,arrows.meta}
%\pgfplotsset{my style/.append style={axis x line=middle, axis y line=
%middle, xlabel={$x$}, ylabel={$y$}, axis equal }}


\begin{document}
\begin{enumerate}
	\item Prove that for every graph $G$, there exists an order of the vertex set of $G$ such that a greedy algorithm using this ordering will use $\chi(G)$ colors.
	\item For every $n \geq 3,$ construct find a bipartite graph on $2n$ vertices and an ordering of the vertex set such that the greedy algorithm will use $n$ colors (as opposed to the optimal 2 colors).
	\item A $k$-chromatic graph $G$ is called \emph{critical} if $\chi(G-v) < k$ for every vertex $v \in G.$
	\begin{enumerate}
	\item Characterize critical $2$-chromatic graphs.
	\item Find an example of a critical $3$-chromatic graph.
	\item Prove that for $k\geq 3$ every critical $k$-chromatic graph is $(k-1)$-edge-connected.
	\item Characterize the set of critical $3$-chromatic graphs.
	\end{enumerate}
	\item The \emph{clique number} of a graph, denoted by $\omega(G),$ is the largest $r$ such that $K^r \subseteq G.$ The \emph{independence number} of a graph, denoted by $\alpha(G),$ is the largest $r$ such that $G$ contains an independent set of vertices of cardinality $r.$
	\begin{enumerate}
	\item Determine $\omega(G)$ and $\alpha(G)$ for:
		\begin{enumerate}
		\item $P^m$ for $m\geq 1$
		\item $C^k$
		\item $K_{m,n}$ where $m \leq n$
		\item $K^n$
		\end{enumerate}
	\item Prove that $\chi(G) \geq \max\{ \omega(G), |G|/\alpha(G)\}.$
	\end{enumerate}
	\item Prove or Disprove: Every $k$-chromatic graph $G$ has a $k$-coloring in which some color class has at least $\alpha(G)$ vertices.
	\item Assume that $H$ is a $k$-chromatic triangle-free graph and the $G$ is obtained from $H$ by Mycielski's Construction.
	\begin{enumerate}
	\item Prove that $G$ is also triangle-free.
	\item Prove that $G$ is $(k+1)$-colorable.
	\end{enumerate}
	\item Describe the topic of your project and what source(s) you have found.
\end{enumerate}
\end{document}